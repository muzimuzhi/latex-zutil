\ProvidesExplFile {zutil-l3extras.code.tex} {2025-06-21} {0.5}
  {Z's utilities, the l3kernel extras module}

%%
%% l3basice extras
%%

\prg_new_conditional:Npnn \zutil_cs_if_function:N #1 { p, T, F, TF }
  {
    \if_bool:N
        \exp_last_unbraced:Nf \use_iii:nnn \cs_split_function:N #1 % noqa: w302
      \prg_return_true: % noqa: w302
    \else:
      \prg_return_false:
    \fi:
  }

\cs_new:Npn \zutil_cs_function_name:N #1
  {
    \exp_last_unbraced:Nf \__zutil_cs_function_name:nnN
      \cs_split_function:N #1 % noqa: w302
  }
\cs_new:Npn \__zutil_cs_function_name:nnN #1#2#3
  {
    \bool_if:NTF #3
      % has colon, function
      { \exp_not:n { #1 } }
      % no colon, variable
      { \exp_not:n { \q_no_value } }
  }

\cs_new:Npn \zutil_cs_function_signature:N #1
  {
    \exp_last_unbraced:Nf \__zutil_cs_function_signature:nnN
      \cs_split_function:N #1 % noqa: w302
  }
\cs_new:Npn \__zutil_cs_function_signature:nnN #1#2#3
  {
    \bool_if:NTF #3
      { \exp_not:n { #2 } }
      { \exp_not:n { \q_no_value } }
  }

\cs_new:Npn \zutil_cs_base_function:N #1
  {
    \exp_last_unbraced:Nf \__zutil_cs_base_function:nnN
      \cs_split_function:N #1 % noqa: w302
  }
\cs_new:Npn \__zutil_cs_base_function:nnN #1#2#3
  {
    \bool_if:NTF #3
      % does \exp_not:N differ from \exp_not:n in any way?
      { \exp_not:c { #1 : \__zutil_cs_sig_base_form:w #2 \q_stop } }
      { \exp_not:n { \q_no_value } }
  }

%%
%% l3prg extras
%%

% Treat \relax as defined, thus different from \cs_if_exist:NTF.
\prg_set_conditional:Npnn \zutil_cs_if_defined:N #1 { p, T, F, TF }
  {
    \if_cs_exist:N #1
      \prg_return_true:
    \else:
      \prg_return_false:
    \fi:
  }
\prg_set_conditional:Npnn \zutil_cs_if_defined:c #1 { p, T, F, TF }
  {
    \if_cs_exist:w #1 \cs_end:
      \prg_return_true:
    \else:
      \prg_return_false:
    \fi:
  }

%%
%% l3expan extras
%%

\msg_new:nnn { zutil } { cs/invalid-arg-spec }
  {
    Invalid~argument~specifier~'#1'.~
    Accepted~specifiers~are~'NcnofexVvpTF'.
  }

% Example:
%     \zutil_cs_generate_variant:N \zutil_set:V
%     \zutil_cs_generate_variant:n { \zutil_set:v, \zutil_set:e }
\cs_new_protected:Npn \zutil_cs_generate_variant:N #1
  {
    \cs_if_exist:NF #1
      {
        % \cs_split_function:N <function> expands to a 3-tuple
        % "{<name:str>} {<signature:str>} <has colon:bool>".
        % As the result starts with "{", possible leading spaces in <name>
        % won't be gobbled by f-type expansion.
        \exp_last_unbraced:Nf \__zutil_cs_generate_variant:nnN
          \cs_split_function:N #1 % noqa: w302
      }
  }
\cs_new_protected:Npn \zutil_cs_generate_variant:n #1
  {
    \clist_map_inline:nn {#1}
      { \zutil_cs_generate_variant:N ##1 }
  }
\cs_new_protected:Npn \__zutil_cs_generate_variant:nnN #1#2#3
  {
    % skip if it's not a function (no colon in csname)
    \bool_if:NT #3
      {
        \tl_set:Ne \l__zutil_tmp_tl
          { \__zutil_cs_sig_base_form:w #2 \q_stop }
        % skip if it's already the base form
        \str_if_eq:onF { \l__zutil_tmp_tl } { #2 }
          {
            % avoid setting undefined base form to \relax
            % a trick learned from https://github.com/latex3/latex2e/commit/ef35caf4324ae6522204917b2da0742ef6a6281f
            \group_begin:
              \exp_after:wN
            \group_end:
              \exp_after:wN
            \cs_generate_variant:Nn
              \cs:w #1: \l__zutil_tmp_tl \cs_end: { #2 }
          }
      }
  }
\cs_new:Npn \__zutil_cs_sig_base_form:w #1
  {
    \cs_if_eq:NNF #1 \q_stop
      {
        \cs_if_exist_use:cF { __zutil_cs_sig_base_form_#1: }
          {
            #1
            \msg_expandable_error:nnn
              { zutil } { cs/invalid-arg-spec } { #1 }
          }
        \__zutil_cs_sig_base_form:w
      }
  }
% space for time. l3doc uses a \str_case:nnTF in
% \@@_signature_base_form_aux:n (<@@=codedoc>)
% Specifiers 'w' and 'D' are _not_ accepted.
\cs_new:Npn \__zutil_cs_sig_base_form_N: { N } % noqa: w401
\cs_new:Npn \__zutil_cs_sig_base_form_c: { N } % noqa: w401
\cs_new:Npn \__zutil_cs_sig_base_form_n: { n } % noqa: w401
\cs_new:Npn \__zutil_cs_sig_base_form_o: { n } % noqa: w401
\cs_new:Npn \__zutil_cs_sig_base_form_f: { n } % noqa: w401
\cs_new:Npn \__zutil_cs_sig_base_form_e: { n } % noqa: w401
\cs_new:Npn \__zutil_cs_sig_base_form_x: { n } % noqa: w401
\cs_new:Npn \__zutil_cs_sig_base_form_V: { n } % noqa: w401
\cs_new:Npn \__zutil_cs_sig_base_form_v: { n } % noqa: w401
\cs_new:Npn \__zutil_cs_sig_base_form_p: { p } % noqa: w401
\cs_new:Npn \__zutil_cs_sig_base_form_T: { T } % noqa: w401
\cs_new:Npn \__zutil_cs_sig_base_form_F: { F } % noqa: w401


%%
%% l3msg extras
%%

% suspend debugging in \msg_<type>:nn(nnnnn) functions
% FIXME: not tested, due to unstable benchmark results
\cs_new_protected:Npn \zutil_msg_suspend_debug:
  {
    \cs_set_eq:NN \__msg_use:nnnnnnn \__zutil_msg_use:nnnnnnn
  }
\cs_new_protected:Npn \zutil_msg_resume_debug:
  {
    \cs_set_eq:NN \__msg_use:nnnnnnn \__zutil_msg_use_orig:nnnnnnn
  }

\cs_gset_eq:NN \__zutil_msg_use_orig:nnnnnnn \__msg_use:nnnnnnn
\cs_new_protected:Npn \__zutil_msg_use:nnnnnnn #1#2#3#4#5#6#7
  {
    \debug_suspend:
    \__zutil_msg_use_orig:nnnnnnn {#1} {#2} {#3} {#4} {#5} {#6} {#7}
    \debug_resume:
  }


%%
%% l3tl extras
%%

% Example
%     \zutil_prg_new_conditional_tl_if_in:Nnn \zutil_if_colon_in:n
%       { : } { TF }
% defines \zutil_if_colon_in:nTF so that
%     \zutil_if_colon_in:nTF {<token list>} {<true code>} {<false code>}
% is an expandable variant of
%     \tl_if_in:nnTF {<token list>} { : } {<true code>} {<false code>}

% #1: func = cs, #2: tl = test-in, #3: clist = conditions
% #2 and #3 are curried
\cs_new_protected:Npn \zutil_prg_new_conditional_tl_if_in:Nnn #1
  {
    \exp_args:Ne \__zutil_prg_new_conditional_tl_if_in:nNnn
      { \exp_last_unbraced:Nf \use_i:nnn \cs_split_function:N #1 } #1 % noqa: w302
  }
% #1: str = name, #2: func = cs, #3: tl = test-in, #4: clist = conditions
\cs_new_protected:Npn \__zutil_prg_new_conditional_tl_if_in:nNnn #1#2#3#4
  {
    \cs_new:cpn { #1_aux : w } ##1 #3 { }
    \prg_new_conditional:Npne #2 ##1 {#4}
      {
        \exp_not:N \tl_if_empty:oTF
          { \exp_not:c { #1_aux : w } ##1 {} {} #3 }
          { \exp_not:N \prg_return_false: }
          { \exp_not:N \prg_return_true: }
      }
  }
\zutil_cs_generate_variant:N \prg_new_conditional:Npne
\cs_generate_variant:Nn \zutil_prg_new_conditional_tl_if_in:Nnn
  { No, NV, Ne }

%%
%% l3seq extras
%%
\tl_new:N \l__zutil_seq_internal_a_tl

\msg_new:nnn { zutil } { seq/empty-delimiter }
  {
    Empty~delimiter~is~not~supported~in~#1. \\
    The~existing~definition~of~'#2'~will~not~be~altered.
  }

% It can be defined as simple as
%     %<@@=seq>
%     \cs_new_protected:Npn \zutil_seq_set_split_keep_braces:Nnn
%       { \@@_set_split:NNNnn \__kernel_tl_set:Nx
%           \__zutil_seq_trim_spaces:n }
%     \cs_new:Npn \__zutil_seq_trim_spaces:n #1
%       { { \tl_trim_spaces:n {#1} } }
% so differs from \seq_set_split:Nnn and \seq_set_split_keep_spaces:Nnn
% by only the space trimming function:
%     \cs_new_protected:Npn \seq_set_split:Nnn
%       { \@@_set_split:NNNnn \__kernel_tl_set:Nx \tl_trim_spaces:n }
%     \cs_new_protected:Npn \seq_set_split_keep_spaces:Nnn
%       { \@@_set_split:NNNnn \__kernel_tl_set:Nx \exp_not:n }
% but I insist on raising an error on empty delimiter.
\cs_new_protected:Npn \zutil_seq_set_split_keep_braces:Nnn #1
  {
    \__zutil_seq_set_split:NNNNnn
      \__kernel_tl_set:Nx \tl_trim_spaces:n #1
      \zutil_seq_set_split_keep_braces:Nnn
  }
\cs_generate_variant:Nn \zutil_seq_set_split_keep_braces:Nnn { NnV }

% gset version
\cs_new_protected:Npn \zutil_seq_gset_split_keep_braces:Nnn #1
  {
    \__zutil_seq_set_split:NNNNnn
      \__kernel_tl_gset:Nx \tl_trim_spaces:n #1
      \zutil_seq_gset_split_keep_braces:Nnn
  }
\cs_generate_variant:Nn \zutil_seq_gset_split_keep_braces:Nnn { NnV }

% Compared to \__seq_set_split:NNNnn, a forth N-arg is added which
% holds the caller, i.e. \zutil_seq_set_split_keep_braces:Nnn,
% for use in error message.
%
% l3seq internals \s__seq and \__seq_item:n are used, which are unavoidable.
\cs_new_protected:Npn \__zutil_seq_set_split:NNNNnn #1#2#3#4#5#6
  {
    \tl_if_empty:nTF {#5}
      {
        \msg_error:nnnn { zutil } { seq/empty-delimiter } {#4} {#3}
      }
      {
        \tl_set:Nn \l__zutil_seq_internal_a_tl
          {
            \__zutil_seq_set_split:Nw #2 \prg_do_nothing:
            #6
            \__zutil_seq_set_split_end:
          }
        \tl_replace_all:Nnn \l__zutil_seq_internal_a_tl {#5}
          {
            \__zutil_seq_set_split_end:
            \__zutil_seq_set_split:Nw #2 \prg_do_nothing:
          }
        % now \l__zutil_seq_internal_a_tl contains a list of
        %     \__zutil_seq_set_split:Nw #2
        %       \prg_do_nothing: <seq item>
        %     \__zutil_set_split_end:
        \__kernel_tl_set:Nx \l__zutil_seq_internal_a_tl
          { \l__zutil_seq_internal_a_tl }
        % now \l__zutil_seq_internal_a_tl contains a list of
        %     \__zutil_seq_wrap_item:n { <seq item> }
        #1 #3 { \s__seq \l__zutil_seq_internal_a_tl }
      }
  }

% This custom set_split:Nw skips the \__seq_set_split:w step which
% would strip a pair of braces.
\cs_new:Npn \__zutil_seq_set_split:Nw #1#2 \__zutil_seq_set_split_end:
  {
    \exp_not:N \__zutil_seq_wrap_item:n { \exp_args:No #1 {#2} }
  }

% Will \__seq_wrap_item:n ever change?
\cs_new:Npn \__zutil_seq_wrap_item:n #1
  {
    \exp_not:n { \__seq_item:n {#1} }
  }
