\ProvidesExplFile {zutil-l3extras.code.tex} {2024-12-14} {0.3}
  {Z's utilities, the l3kernel extras module}

%%
%% l3prg extras
%%

% Treat \relax as defined, thus different from \cs_if_exist:NTF.
\prg_set_conditional:Npnn \__zutil_cs_if_defined:N #1 { p, T, F, TF }
  {
    \if_cs_exist:N #1
      \prg_return_true:
    \else:
      \prg_return_false:
    \fi:
  }
\prg_set_conditional:Npnn \__zutil_cs_if_defined:c #1 { p, T, F, TF }
  {
    \if_cs_exist:w #1 \cs_end:
      \prg_return_true:
    \else:
      \prg_return_false:
    \fi:
  }

%%
%% l3tl extras
%%

% To better understand how \tl_trim_spaces:n works, consult latex3 commits 
% dc98b75d (\tl_trim_spaces:n trimming recursively, only explicit spaces.,
% 2011-08-13)
% https://github.com/latex3/latex3/commit/dc98b75d4d070608aaf387a4abd543e0ad8dc0cd
% 5c3f4ff3 (slightly faster tl_trim_spaces:n, 2020-08-09)
% https://github.com/latex3/latex3/commit/5c3f4ff31ee97f4ef81872ddbc7b6f2eb1fd5cff

% \zutil_tl_trim_leading_spaces:n {<tokens>}
% \zutil_tl_trim_trailing_spaces:n {<tokens>}
%
% similar to \tl_trim_spaces:n except the \__zutil_tl_trim_mark: is moved
% in \__zutil_tl_trim_(leading|trailing)_spaces:nn
\cs_new:Npn \zutil_tl_trim_leading_spaces:n #1
  {
    \__zutil_tl_trim_leading_spaces:nn
      { #1 }
      { \__kernel_exp_not:w \exp_after:wN }
  }
\cs_generate_variant:Nn \zutil_tl_trim_leading_spaces:n { V , v , e , o }
\cs_new:Npn \zutil_tl_trim_trailing_spaces:n #1
  {
    \__zutil_tl_trim_trailing_spaces:nn
      { #1 }
      { \__kernel_exp_not:w \exp_after:wN }
  }
\cs_generate_variant:Nn \zutil_tl_trim_trailing_spaces:n { V , v , e , o }

% \zutil_tl_trim_leading_spaces_apply:nN { <tokens> } { <func:n> }
% \zutil_tl_trim_trailing_spaces_apply:nN { <tokens> } { <func:n> }
%
% \tl_trim_spaces_apply:nN doesn't wrap space-trimming result of its #1
% within \exp_not:n, thus is a bit different from \tl_trim_spaces:n and
% we provide single-side versions for it.
\cs_new:Npn \zutil_tl_trim_leading_spaces_apply:nN #1#2
  { \__zutil_tl_trim_leading_spaces:nn { #1 } { \exp_args:No #2 } }
\cs_generate_variant:Nn \zutil_tl_trim_leading_spaces_apply:nN { o }
\cs_new:Npn \zutil_tl_trim_trailing_spaces_apply:nN #1#2
  { \__zutil_tl_trim_trailing_spaces:nn { #1 } { \exp_args:No #2 } }
\cs_generate_variant:Nn \zutil_tl_trim_trailing_spaces_apply:nN { o }

% \tl_trim_spaces:N and \tl_gtrim_spaces:N can be easily emulated using
% \tl_(g)set:Ne.

% The implementation of \tl_trim_spaces:nn in l3kernel uses these internals:
% - \__tl_trim_spaces:nn { \__tl_trim_mark: <tokens> } { <continuation> }
%   adds needed tokens for recursive application of auxi:w and
%   auxiii:w, and auxiv:w
% - \__tl_trim_spaces_auxi:w removes a leading space from <tokens>
% - \__tl_trim_spaces_auxii:w glues auxi:w and auxiii:w
% - \__tl_trim_spaces_auxiii:w removes a trailing space from <tokens>
% - \__tl_trim_spaces_auxiv:w absorb and apply <continuation>
% - \__tl_trim_mark: marker prepended to <tokens> to prevent brace
%   stripping and as part of the delimiters of auxi:w; equal to
%   \prg_do_nothing:
% - \s__tl_nil scan marker appended to <tokens> to prevent brace
%   stripping and as part of the delimiters of auxiii:w
%
% We reuse \s__tl_nil, define aliases for (using "zutil" module name)
% - \__zutil_tl_trim_spaces_auxi:w
% - \__zutil_tl_trim_spaces_auxii:w (not in use)
% - \__zutil_tl_trim_spaces_auxiii:w
% - \__zutil_tl_trim_spaces_auxiv:w
% - \__zutil_tl_trim_mark:
% and define new internals
% - \__zutil_tl_trim_leading_spaces:nn { <tokens> } { <continuation> }
% - \__zutil_tl_trim_trailing_spaces:nn { <tokens> } { <continuation> }
% - \__zutil_tl_trim_spaces_auxv:w glues auxi:w and auxiv:w

\group_begin:
  \cs_set_protected:Npn \__zutil_tmp:w #1
    {
      \cs_new:Npn \__zutil_tl_trim_leading_spaces:nn ##1
        {
          \__zutil_tl_trim_spaces_auxi:w
            \__zutil_tl_trim_mark: ##1 \s__tl_nil
            \__zutil_tl_trim_mark: #1 { }
            \__zutil_tl_trim_mark: \__zutil_tl_trim_spaces_auxv:w
          \s__tl_stop
        }

      \cs_new:Npn \__zutil_tl_trim_trailing_spaces:nn ##1
        {
          \__zutil_tl_trim_spaces_auxiii:w
            \__zutil_tl_trim_mark: ##1 \s__tl_nil
            \__zutil_tl_trim_spaces_auxiii:w
            #1 \s__tl_nil
            \__zutil_tl_trim_spaces_auxiv:w
          \s__tl_stop
        }

      % the same as \__tl_trim_spaces_auxi:w
      \cs_new:Npn \__zutil_tl_trim_spaces_auxi:w
          ##1 \__zutil_tl_trim_mark: #1 ##2 \__zutil_tl_trim_mark: ##3
        {
          ##3
          \__zutil_tl_trim_spaces_auxi:w
          \__zutil_tl_trim_mark: ##2
          \__zutil_tl_trim_mark: #1 {##1}
        }

      % the same as \__tl_trim_spaces_auxiii:w
      \cs_new:Npn \__zutil_tl_trim_spaces_auxiii:w ##1 #1 \s__tl_nil ##2
        {
          ##2
          ##1 \s__tl_nil
          \__zutil_tl_trim_spaces_auxiii:w
        }
    }
  \__zutil_tmp:w { ~ }
\group_end:

% \__tl_trim_spaces_auxi:w/\__zutil_tl_trim_spaces_auxi:w explained
%
% when <tokens> contain a leading space (thus "\__zutil_tl_trim_mark: ~"
% matches it), auxi:w expands to a recursion with
% #1 = "", #2 = "<tokens'>" \s__tl_nil", #3 = ""
%     \__zutil_tl_trim_spaces_auxi:w
%       \__zutil_tl_trim_mark: <tokens'> \s__tl_nil
%       \__zutil_tl_trim_mark: ~ { }
%       \__zutil_tl_trim_mark: \__zutil_tl_trim_spaces_auxv:w
%     \s__tl_stop
%     {<continuation>}
% otherwise, auxi:w expands to auxii:w or auxv:w with
% #1 = "\__zutil_tl_trim_mark: <tokens> \s__tl_nil", #2 = "",
% #3 = <next aux>, auxii:w or auxv:w
%     \__zutil_tl_trim_spaces_auxv:w
%       \__zutil_tl_trim_spaces_auxi:w
%       \__zutil_tl_trim_mark:
%       \__zutil_tl_trim_mark: ~ { \__zutil_tl_trim_mark: <tokens> \s__tl_nil }
%     \s__tl_stop
%     {<continuation>}

% \__tl_trim_spaces:auxiii:w/\__zutil_tl_trim_spaces_auxiii:w explained
%
% when <tokens> contain a trailing space (thus "~ \s__tl_nil" matches it),
% auxiii:w expands to a recursion with
% #1 = "\__zutil_tl_trim_mark: <tokens'>", #2 = auxiii:w
    % \__zutil_tl_trim_spaces_auxiii:w
    %   \__zutil_tl_trim_mark: <tokens'> \s__tl_nil
    %   \__zutil_tl_trim_spaces_auxiii:w
    %   ~ \s__tl_nil
    %   \__zutil_tl_trim_spaces_auxiv:w
    % \s__tl_stop
    % {<continuation>}
% otherwise, auxiii:w expands to auxiv:w with
% #1 = "\__zutil_tl_trim_mark: #1 \s__tl_nil
%       \__zutil_tl_trim_spaces_auxiii:w",
% #2 = auxiv:w
%     \__zutil_tl_trim_spaces_auxiv:w
%       \__zutil_tl_trim_mark: #1 \s__tl_nil
%       \__zutil_tl_trim_spaces_auxiii:w
%       \__zutil_tl_trim_spaces_auxiv:w
%     \s__tl_stop
%     {<continuation>}

% the same as \__tl_trim_spaces_auxii:w
\cs_new:Npn \__zutil_tl_trim_spaces_auxii:w
    \__zutil_tl_trim_spaces_auxi:w
    \__zutil_tl_trim_mark: \__zutil_tl_trim_mark: #1
  {
    \__zutil_tl_trim_spaces_auxiii:w
    #1
  }

% the same as \__tl_trim_spaces_auxiv:w
\cs_new:Npn \__zutil_tl_trim_spaces_auxiv:w
    #1 \s__tl_nil #2 \s__tl_stop #3
  { #3 { #1 } }

% similar to \__tl_trim_spaces_auxii:w
%
% auxii:w glues
%   auxi:w (leading space trimmer) and auxiii:w (trailing space trimmer),
% while zutil..._auxv:w glues
%   zutil...auxi:w (leading space trimmer) and zutil...auxiv:w (ending)
\cs_new:Npn \__zutil_tl_trim_spaces_auxv:w
    \__zutil_tl_trim_spaces_auxi:w
    \__zutil_tl_trim_mark: \__zutil_tl_trim_mark: #1
  {
    \__zutil_tl_trim_spaces_auxiv:w
    #1
  }

\cs_new:Npn \__zutil_tl_trim_mark: {}

%%
%% l3seq extras
%%
\tl_new:N \l__zutil_seq_internal_a_tl

\msg_new:nnn { zutil } { seq/empty-delimiter }
  {
    Empty~delimiter~is~not~supported~in~#1. \\
    The~existing~definition~of~'#2'~will~not~be~altered.
  }

% It can be defined as simple as
%     %<@@=seq>
%     \cs_new_protected:Npn \zutil_seq_set_split_keep_braces:Nnn
%       { \@@_set_split:NNNnn \__kernel_tl_set:Nx
%           \__zutil_seq_trim_spaces:n }
% so differs from \seq_set_split:Nnn and \seq_set_split_keep_spaces:Nnn
% by only the space trimming function:
%     \cs_new_protected:Npn \seq_set_split:Nnn
%       { \@@_set_split:NNNnn \__kernel_tl_set:Nx \tl_trim_spaces:n }
%     \cs_new_protected:Npn \seq_set_split_keep_spaces:Nnn
%       { \@@_set_split:NNNnn \__kernel_tl_set:Nx \exp_not:n }
% but I insist on raising an error on empty delimiter.
\cs_new_protected:Npn \zutil_seq_set_split_keep_braces:Nnn #1
  {
    \__zutil_seq_set_split:NNNNnn
      \__kernel_tl_set:Nx \tl_trim_spaces:n #1
      \zutil_seq_set_split_keep_braces:Nnn
  }
\cs_generate_variant:Nn \zutil_seq_set_split_keep_braces:Nnn { NnV }

% gset version
\cs_new_protected:Npn \zutil_seq_gset_split_keep_braces:Nnn #1
  {
    \__zutil_seq_set_split:NNNNnn
      \__kernel_tl_gset:Nx \tl_trim_spaces:n #1
      \zutil_seq_gset_split_keep_braces:Nnn
  }
\cs_generate_variant:Nn \zutil_seq_gset_split_keep_braces:Nnn { NnV }

% % This implementation is quicker than adding another \prg_do_nothing:
% % and \exp_args:No trick to \__zutil_seq_set_split:Nw and
% % \__zutil_seq_set_split:w, resp.
% \cs_new:Npn \__zutil_seq_trim_spaces:n #1 { { \tl_trim_spaces:n {#1} } }

% Compared to \__seq_set_split:NNNnn, a forth N-arg is added which
% holds the caller, i.e. \zutil_seq_set_split_keep_braces:Nnn,
% for use in error message.
%
% l3seq internals \s__seq and \__seq_item:n are used, which are unavoidable.
\cs_new_protected:Npn \__zutil_seq_set_split:NNNNnn #1#2#3#4#5#6
  {
    \tl_if_empty:nTF {#5}
      {
        \msg_error:nnnn { zutil } { seq/empty-delimiter } {#4} {#3}
      }
      {
        \tl_set:Nn \l__zutil_seq_internal_a_tl
          {
            \__zutil_seq_set_split:Nw #2 \prg_do_nothing:
            #6
            \__zutil_seq_set_split_end:
          }
        \tl_replace_all:Nnn \l__zutil_seq_internal_a_tl {#5}
          {
            \__zutil_seq_set_split_end:
            \__zutil_seq_set_split:Nw #2 \prg_do_nothing:
          }
        % now \l__zutil_seq_internal_a_tl contains a list of
        %   \__zutil_seq_set_split:Nw #2
        %     \prg_do_nothing: <seq item>
        %   \__zutil_set_split_end:
        \__kernel_tl_set:Nx \l__zutil_seq_internal_a_tl { \l__zutil_seq_internal_a_tl }
        % now \l__zutil_seq_internal_a_tl contains a list of
        %   \__zutil_seq_wrap_item:n { <seq item> }
        #1 #3 { \s__seq \l__zutil_seq_internal_a_tl }
      }
  }

\cs_new:Npn \__zutil_seq_set_split:Nw #1#2 \__zutil_seq_set_split_end:
  {
    \exp_not:N \__zutil_seq_wrap_item:n { \exp_args:No #1 {#2} }
  }

\cs_new:Npn \__zutil_seq_wrap_item:n #1
  {
    \exp_not:n { \__seq_item:n {#1} }
  }
