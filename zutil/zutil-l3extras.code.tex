\ProvidesExplFile {zutil-l3extras.code.tex} {2024-12-14} {0.3}
  {Z's utilities, the l3kernel extras module}

%%
%% l3seq extras
%%
\msg_new:nnn { zutil } { seq/empty-delimiter }
  {
    Empty~delimiter~is~not~supported~in~#1. \\
    The~existing~definition~of~'#2'~will~not~be~altered.
  }

% It can be defined as simple as
%     %<@@=seq>
%     \cs_new_protected:Npn \zutil_seq_set_split_keep_braces:Nnn
%       { \@@_set_split:NNNnn \__kernel_tl_set:Nx
%           \__zutil_seq_trim_spaces:n }
% so differs from \seq_set_split:Nnn and \seq_set_split_keep_spaces:Nnn
% by only the space trimming function:
%     \cs_new_protected:Npn \seq_set_split:Nnn
%       { \@@_set_split:NNNnn \__kernel_tl_set:Nx \tl_trim_spaces:n }
%     \cs_new_protected:Npn \seq_set_split_keep_spaces:Nnn
%       { \@@_set_split:NNNnn \__kernel_tl_set:Nx \exp_not:n }
% but I insist on raising an error on empty delimiter.
\cs_new_protected:Npn \zutil_seq_set_split_keep_braces:Nnn #1
  {
    \__zutil_seq_set_split:NNNNnn
      \__kernel_tl_set:Nx \__zutil_seq_trim_spaces:n #1
      \zutil_seq_set_split_keep_braces:Nnn
  }
\cs_generate_variant:Nn \zutil_seq_set_split_keep_braces:Nnn { NnV }

% gset version
\cs_new_protected:Npn \zutil_seq_gset_split_keep_braces:Nnn #1
  {
    \__zutil_seq_set_split:NNNNnn
      \__kernel_tl_gset:Nx \__zutil_seq_trim_spaces:n #1
      \zutil_seq_gset_split_keep_braces:Nnn
  }
\cs_generate_variant:Nn \zutil_seq_gset_split_keep_braces:Nnn { NnV }

% This implementation is quicker than adding another \prg_do_nothing:
% and \exp_args:No trick to \__zutil_seq_set_split:Nw and
% \__zutil_seq_set_split:w, resp.
\cs_new:Npn \__zutil_seq_trim_spaces:n #1 { { \tl_trim_spaces:n {#1} } }

% Compared to \__seq_set_split:NNNnn, a forth N-arg is added which
% holds the caller, i.e. \zutil_seq_set_split_keep_braces:Nnn,
% for use in error message.
%
% l3seq internals \__seq_set_split:Nw and \l__seq_internal_a_tl, along
% with the undefined delimiter \__seq_set_split_end:, are used.
\cs_new_protected:Npn \__zutil_seq_set_split:NNNNnn #1#2#3#4#5#6
  {
    \tl_if_empty:nTF {#5}
      {
        \msg_error:nnnn { zutil } { seq/empty-delimiter } {#4} {#3}
      }
      {
        \tl_set:Nn \l__seq_internal_a_tl
          {
            \__seq_set_split:Nw #2 \prg_do_nothing:
            #6
            \__seq_set_split_end:
          }
        \tl_replace_all:Nnn \l__seq_internal_a_tl {#5}
          {
            \__seq_set_split_end:
            \__seq_set_split:Nw #2 \prg_do_nothing:
          }
        \__kernel_tl_set:Nx \l__seq_internal_a_tl { \l__seq_internal_a_tl }
        #1 #3 { \s__seq \l__seq_internal_a_tl }
      }
  }

%%
%% l3prg extras
%%

% Treat \relax as defined, thus different from \cs_if_exist:NTF.
\prg_set_conditional:Npnn \__zutil_cs_if_defined:N #1 { p, T, F, TF }
  {
    \if_cs_exist:N #1
      \prg_return_true:
    \else:
      \prg_return_false:
    \fi:
  }
\prg_set_conditional:Npnn \__zutil_cs_if_defined:c #1 { p, T, F, TF }
  {
    \if_cs_exist:w #1 \cs_end:
      \prg_return_true:
    \else:
      \prg_return_false:
    \fi:
  }
