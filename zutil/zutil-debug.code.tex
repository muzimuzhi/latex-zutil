\ProvidesExplFile {zutil-debug.code.tex} {2024-11-21} {0.2}
  {Z's\space utilities\space for\space debugging}

% TODO: do not load by default
% TODO: choose how debug info is used, log, term, collect

% TBD: switch arg order of \zutil_debug:nn?

%%
%% named variables
%%

% label keys
\seq_new:N   \l__zutil_debug_labels_seq
% level keys
\int_new:N   \l__zutil_debug_level_int
% if key
\bool_new:N  \l__zutil_debug_do_bool
\bool_set_true:N \l__zutil_debug_do_bool
% if decorators
\prop_new_linked:N \g__zutil_debug_collectors_prop

% variants
\cs_generate_variant:Nn \prop_gput_from_keyval:Nn { Ne }

%%
%% argument collectors
%%

% wrapper around \cs_generate_variant:Nn which also adds collectors for
% variants
%
% @remarks
% Every arg spec in #2 (variant arg list) MUST be full. Usage
%   \zutil_debug_generate_variant:Nn \zutil_debug:nN { e }
% is INVALID but without errors.
\cs_new_protected:Npn \zutil_debug_generate_variant:Nn #1#2
  {
    \cs_generate_variant:Nn #1 { #2 }
    % stores the cs name of base form #1, without colon and signature
    \str_set:Ne \l__zutil_tmp_str
      {
        \exp_last_unbraced:Nf \use_i:nnn
          \cs_split_function:N #1
      }
    % stores the collector of base form #1
    \tl_set:Ne \l__zutil_tmp_tl
      { \__zutil_debug_collect_args:Nw #1 }
    \clist_map_inline:nn { #2 }
      {
        \__zutil_debug_gput_collector:co
          { \l__zutil_tmp_str : ##1 }
          { \l__zutil_tmp_tl }
      }
  }

\IfExplAtLeastTF{2024-12-07}
  {
    % #1 and #2 are curried
    \cs_new_protected:Npn \__zutil_debug_gput_collector:Nn
      {
        \prop_gput:Nnn \g__zutil_debug_collectors_prop
      }
  }
  {
    % work around a l3kernel bug latex3/latex3#1630
    \cs_new_protected:Npn \__zutil_debug_gput_collector:Nn #1#2
      {
        \prop_gput:Nnn \g__zutil_debug_collectors_prop
          {#1} { \exp_not:n {#2} }
      }
  }
\cs_generate_variant:Nn \__zutil_debug_gput_collector:Nn { co }

% all args are curried
\cs_new:Npn \__zutil_debug_collect_args:Nw
  {
    \prop_item:Nn \g__zutil_debug_collectors_prop
  }

%%
%% main debug command
%%

% IDEA: if \iow_wrap:nnnN is needed, then this would be more
%       like \tl_log:n. Also note \tl_show:n uses \tex_showtokens:D.
% hmm this gets more complicated
% #1 : keyval = options, #2 : tl = debug text
\cs_new_protected:Npn \zutil_debug:nn #1#2
  {
    \keys_set_groups:nnn { zutil/debug } { global } { #1 }
    \group_begin:
    \keys_set_exclude_groups:nnn { zutil/debug } { config, global } { #1 }
    \bool_if:NT \l__zutil_debug_do_bool
      {
        % if seq is empty, tl is set to (both contain and equal to)
        % \q_no_value
        \seq_pop_left:NN \l__zutil_debug_labels_seq \l__zutil_tmp_tl
        \zutil_debug_do:e
          {
            % pretend it's an error, so can be filtered out by TeXstudio
            !~debug~
            \tl_if_eq:NNF \l__zutil_tmp_tl \q_no_value
              {
                % output forms
                % [label1]
                % [label1]label2/label3/.../labeln
                [ \l__zutil_tmp_tl ]
                \seq_use:Nn \l__zutil_debug_labels_seq {/} ~
              }
            % TBD: which way is better/quicker, \exp_not:n or
            %      \exp_args:No \cmd { \tl_to_str:n {#1} }?
            \zutil_debug_use_text:n { \exp_not:n {#2} }
            \iow_newline:
          }
      }
    \group_end:
  }
\__zutil_debug_gput_collector:Nn \zutil_debug:nn { \use_none:nn }
\zutil_debug_generate_variant:Nn \zutil_debug:nn { ne, en, ee }

% #1 : tl = expandable formatted labels
% #2 : int = level
% #3 : tl = expandable formatted text
\cs_new:Npn \zutil_debug_use_level:nNn #1#2#3
  {
    #1
    \prg_replicate:nn {#2} { \c_space_tl \c_space_tl }
    #3
  }

% centric public toggle
% \iow_term:n writes to both terminal and log; \iow_log:n writes to log only
\cs_new_eq:NN \zutil_debug_do:n \iow_term:n
\cs_generate_variant:Nn \zutil_debug_do:n { e }
% \cs_new_eq:NN \zutil_debug_do:n \use_none:n % turn off

% \cs_new:Npn \zutil_debug_use_label:n #1
%   { [#1] }
\cs_new:Npn \zutil_debug_use_text:n #1
  {
    \prg_replicate:nn {\l__zutil_debug_level_int}
      { \c_space_tl \c_space_tl } >>#1<<
  }

%%
%% key-value processing
%%

% - \zutil_debug:nn should accept as few pre-defined keys as possible
% - \zutiL_debug_set:n may accept a bit more keys

% #1 is curried
\cs_new_protected:Npn \zutil_debug_set:n
  {
    \keys_set:nn { zutil/debug }
  }

\keys_define:nn { zutil/debug }
  {
    % label keys
    label .code:n   = \__zutil_debug_set_label:n {#1}
  , label .value_required:n = true
  , e     .code:n   = \exp_args:Ne \__zutil_debug_set_label:n {#1}
  , e     .value_required:n = true
    % level keys
  , +     .code:n   = \__zutil_debug_incr_level:
  , +     .value_forbidden:n = true
  , +     .groups:n = { global }
  , -     .code:n   = \__zutil_debug_decr_level:
  , -     .groups:n = { global }
  , -     .value_forbidden:n = true
  , +.    .code:n   = \__zutil_debug_incr_level:
  , +.    .value_forbidden:n = true
  , -.    .code:n   = \__zutil_debug_decr_level:
  , -.    .value_forbidden:n = true
    % the "if" key
  , if    .code:n   =
    {
      \__zutil_debug_if_skip:nT {#1}
        { \bool_set_false:N \l__zutil_debug_do_bool }
    }
  , if    .value_required:n = true
    % unknown keys are treated as string labels and their values are
    % always dropped, if any.
    % pity "unknown .value_forbidden:n = true" doesn't work
  , unknown .code:n =
      \exp_args:No \__zutil_debug_set_label:n { \l_keys_key_str }
    % keys available to \zutil_debug_set:n only
  , reset-labels .code:n =
    {
      \seq_clear:N \l__zutil_debug_labels_seq
    }
  , reset-labels .value_forbidden:n = true
  , reset-labels .groups:n = { config }
  , reset-label  .meta:n = { reset-labels }
  , reset-label  .value_forbidden:n = true
  , reset-label  .groups:n = { config }
  , reset-level  .code:n = \int_zero:N \l__zutil_debug_level_int
  , reset-level  .groups:n = { config }
  , reset-level  .value_forbidden:n = true
  }

% process label keys
\cs_new_protected:Npn \__zutil_debug_set_label:n
  {
    \seq_put_right:Nn \l__zutil_debug_labels_seq
  }

% process level keys
\cs_new_protected:Npn \__zutil_debug_incr_level:
  {
    \int_incr:N \l__zutil_debug_level_int
  }

\msg_new:nnn { zutil } { debug/negative-level }
  {
    Invalid~negative~debugging~level. \\
    Level~will~be~set~to~zero.
  }
\cs_new_protected:Npn \__zutil_debug_decr_level:
  {
    \int_decr:N \l__zutil_debug_level_int
    \int_compare:nNnT \l__zutil_debug_level_int < 0
      {
        \msg_error:nn { zutil } { debug/negative-level }
        \int_zero:N \l__zutil_debug_level_int
      }
  }

%%
%% if decorators
%%

% If decorators are used as a prefix/decorator of \zutil_debug_if:nn and
% friends, so can be inserted and commented out independently.
%
% When <fp expr> evaluates to "false" (any non-zero value), an if
% decorator gobbles the next debugging command along with its arguments.
% \zutil_debug_if:n requires the following token is one of the debugging
% commands, while \zutil_debug_safe_if:n skips itself if the next token
% is not one of debugging commands, thus is safer.
%
% Example:
%   \zutil_debug_if:n {<fp expr>}
%   \zutil_debug:n {<text>}
%
% @remarks
% If decorators don't follow the expl3 naming convension. If so they
% should have "nNw" or "nw" arg-spec.
\cs_new:Npn \zutil_debug_if:n #1
  {
    \__zutil_debug_if_skip:nT {#1}
      { \__zutil_debug_collect_args:Nw }
  }

% #2 is curried
\cs_new:Npn \__zutil_debug_if_skip:nT #1
  {
    % treat anything non-zero as "true" (means do debug)
    % paring \c_zero_fp is quicker than parsing fp operator "false"
    \fp_compare:nNnT {#1} = { \c_zero_fp }
  }

\cs_new_protected:Npn \zutil_debug_safe_if:n #1
  {
    \__zutil_debug_if_skip:nT {#1}
      % NOTE: use "\peek_catcode:NT \l_peek_token {...}" if
      %       \group_align_safe_begin: is needed
      { \peek_after:Nw \__zutil_debug_safe_collect_args:w }
  }

\cs_new_protected:Npn \__zutil_debug_safe_collect_args:w
  {
    \token_if_cs:NT \l_peek_token
      { \__zutil_debug_safe_collect_args:Nw }
  }

\cs_new:Npn \__zutil_debug_safe_collect_args:Nw #1
  {
    \exp_args:NNe \__zutil_debug_safe_collect_args:Nnw #1
      { \__zutil_debug_collect_args:Nw #1 }
  }

% #1 = \zutil_debug... func., #2: collector
\cs_new:Npn \__zutil_debug_safe_collect_args:Nnw #1#2
  {
    % if #1 is not a debug function, #2 is empty and put #1 back to the
    % input stream
    \tl_if_empty:nTF {#2} {#1} {#2}
  }

%%
%% more debug commands
%%

% :nN and its variants
% nN, nc, eN, ec forms
\cs_new_protected:Npn \zutil_debug:nN #1#2
  {
    \zutil_debug:ne {#1} { \__zutil_debug_use:N #2 }
  }
\cs_new_protected:Npn \zutil_debug:nc #1#2
  {
    \zutil_debug:ne {#1} { \__zutil_debug_use:c {#2} }
  }
\__zutil_debug_gput_collector:Nn \zutil_debug:nN { \use_none:nn }
\__zutil_debug_gput_collector:Nn \zutil_debug:nc { \use_none:nn }
\zutil_debug_generate_variant:Nn \zutil_debug:nN { eN }
\zutil_debug_generate_variant:Nn \zutil_debug:nc { ec }

\cs_new:Npn \__zutil_debug_use:N #1
  {
    \__zutil_cs_if_defined:NTF #1
      { \cs_replacement_spec:N #1 } { \tl_to_str:n { undefined } }
  }
\cs_new:Npn \__zutil_debug_use:c #1
  {
    \__zutil_cs_if_defined:cTF {#1}
      { \cs_replacement_spec:c {#1} } { \tl_to_str:n { undefined } }
  }

% :n, :N and their variations, taking only the text
% n, e, N, c
\cs_new_protected:Npn \zutil_debug:n { \zutil_debug:nn {} }
\__zutil_debug_gput_collector:Nn \zutil_debug:n { \use_none:n }
\zutil_debug_generate_variant:Nn \zutil_debug:n { e }

\cs_new_protected:Npn \zutil_debug:N { \zutil_debug:nN {} }
\cs_new_protected:Npn \zutil_debug:c { \zutil_debug:nc {} }
\__zutil_debug_gput_collector:Nn \zutil_debug:N { \use_none:n }
\__zutil_debug_gput_collector:Nn \zutil_debug:c { \use_none:n }

%%
%% LaTeX2e interfaces
%%

% pros: shorter names
% cons: harder expansion indication (no c, e-type)

% Examples
%     \ZutilDebug[label=x]{msg}
%     \ExpandArgs{e}\ZutilDebug{\value{page}}[label=x]
%     \ExpandArgs{c}\ZutilDebugCmd{__tblr_...}
\NewDocumentCommand \ZutilDebug { O{} m O{} }
  {
    \zutil_debug:nn {#1,#3} {#2}
  }

\NewDocumentCommand \ZutilDebugCmd { O{} m O{} }
  {
    \zutil_debug:nN {#1,#3} #2
  }

%%
%% tabularray support
%%

\cs_new_protected:Npn \__zutil_debug_tblr_support:
  {
    \keys_define:nn { zutil/debug }
      {
        % expansion is deferred until the label is actually being used
        tblr .meta:n = { label = { \the\c@rownum , \the\c@colnum } }
      }
  }
\IfPackageLoadedTF { tabularray }
  { \__zutil_debug_tblr_support: }
  {
    % "package/.../before" hook also works
    \hook_gput_code:nnn { package/tabularray/after } { zutil/debug }
      { \__zutil_debug_tblr_support: }
  }
