
%%% --------------------------------------------------------
%%% for tex4ht-4ht.tex to update tabularray.4ht file
%%% --------------------------------------------------------
%%% need to remove line "\Hinput{tabularray}" in tabularray.4ht 

\<tabularray.4ht\><<< 
% tabularray.4ht (|version), generated from |jobname.tex 
% Copyright 2022 TeX Users Group 
|<TeX4ht license text|> 
|<tabularray definitions|>
|<tabularray table and rows|>
|<tabularray lines|>
|<tabularray cell|>
\Hinput{tabularray} 
\endinput
>>> \AddFile{9}{tabularray}

\<tabularray definitions\><<<
\UseTblrLibrary{hook,html}
\NewConfigure{tabularray}{8}
\NewConfigure{tabularrayignoredcell}{2}
\NewConfigure{tabularrayattributes}[1]{\concat:config\CellAttributes{#1\space}}
\NewConfigure{tabularraystyles}[1]{\concat:config\CellStyle{#1}}
\NewConfigure{tabularraycolumnwidth}{1}

\ExplSyntaxOn
\NewConfigure{tabularrayhalign}[2]{%
  \cs_set:cpn{tabularray_halign:#1}{#2}
}

\NewConfigure{tabularrayvalign}[2]{%
  \cs_set:cpn{tabularray_valign:#1}{#2}
}
\ExplSyntaxOff
>>>

Insert basic tags for table and rows

\<tabularray table and rows\><<<
\ExplSyntaxOn
\AddToHook{tabularray/table/before}{\a:tabularray}
\AddToHook{tabularray/table/after}{\b:tabularray}

\AddToHook{tabularray/row/before}{\xdef\HRow{\@arabic\c@rownum}\c:tabularray}
\AddToHook{tabularray/row/after}{\d:tabularray}
\ExplSyntaxOff
>>>

Tabularray uses hrules in many places, it is not practical to redefine all commands that contain them,
se we disable underscores that are produced for rules using special.

\<tabularray lines\><<<
% disable rules
\ht:special{t4ht@_}
>>>

We detect borders for each cell using these functions. They can be then set using CSS.

Tabularray supports multiple rules between rows and cells, but these would be too difficult to support
using CSS, so we don't try that.

\<tabularray lines\><<<
\ExplSyntaxOn
% #1 tabularray border position, #2 css property to be set
\def\:tblr:linestyle#1#2{
  % create CSS only when a dash style is set 
  \tl_if_empty:cF {lTblrCell#1BorderStyleTl}{
    \def\:hlinecolor{\#000000}
    % convert color to CSS value if color is set
    \tl_if_empty:cF{lTblrCell#1BorderColorTl}{
      \get:xcolorcss{\use:c{lTblrCell#1BorderColorTl}}\:hlinecolor
    }
    % \Configure{tabularraystyles} doesn't expand attributes, so we need to expand it here
    % otherwise, we would get wrong color and hline style in the last row, because this macro is called twice here
    \cs_set:cx{#2:}{
      #2:\dim_to_decimal_in_unit:nn{\use:c{lTblrCell#1BorderWidthTl}*2}{1px}px~
      \use:c{lTblrCell#1BorderStyleTl}\space~\:hlinecolor;
    }
    \Configure{tabularraystyles}{\csname#2:\endcsname}
  }
}
\ExplSyntaxOff
>>>

The code for cell is a bit complicated. We need to construct CSS properties
for horizontal and vertical alignment, background color and rules. 

\<tabularray cell\><<<
\ExplSyntaxOn
\AddToHook{tabularray/cell/before}{
  \xdef\HCol{\@arabic\c@colnum}
  \xdef\HRow{\@arabic\c@rownum}
  \xdef\HMultispan{\lTblrCellColSpanTl}
  \xdef\HRowspan{\lTblrCellRowSpanTl}
  \let\CellAttributes\@empty
  \let\CellStyle\@empty
  \cs_if_exist_use:c{tabularray_halign:\g__tblr_cell_halign_tl}
  \cs_if_exist_use:c{tabularray_valign:\g__tblr_cell_valign_tl}
  % the vertical alignment can be set also in \g__tblr_cell_middle_tl, so we should try it as well
  \cs_if_exist_use:c{tabularray_valign:\g__tblr_cell_middle_tl}
  % calculate column width
  \dim_compare:nNnT {\__tblr_data_item:nen{column}{\HCol}{@col-width}} > {0pt}{
    \__tblr_get_table_width:% initialize \tablewidth
    \edef\HColWidth{\fp_eval:n{\__tblr_data_item:nen{column}{\HCol}{@col-width}/\tablewidth*100}\%}
    % save table width, preferably in CSS
    \a:tabularraycolumnwidth%
  }
  % support for hlines and vlines
  \:tblr:linestyle{Above}{border-top}
  \:tblr:linestyle{Below}{border-bottom}
  \:tblr:linestyle{Left}{border-left}
  \:tblr:linestyle{Right}{border-right}
  % support for the background color
  % save background color to the list of CSS, if it is set
  \tl_if_empty:NF \lTblrCellBackgroundTl{
    \get:xcolorcss{\lTblrCellBackgroundTl}\:bgcolor
    \Configure{tabularraystyles}{background-color: \:bgcolor;}
  }
  % We can use something like \Configure{tabularrayattributes}{rowspan="\HRowspan"} in \Configure{tabularray}
  % to declare correct attributes for joined cells
  \int_compare:nNnT {\HRowspan} > {1}{\g:tabularray}
  \int_compare:nNnT {\HMultispan} > {1}{\h:tabularray}
  \bool_if:NT \lTblrCellOmittedBool {\a:tabularrayignoredcell}
  \e:tabularray
}
\AddToHook{tabularray/cell/after}{
  \f:tabularray
  \bool_if:NT \lTblrCellOmittedBool {\b:tabularrayignoredcell}
}

\ExplSyntaxOff
>>>

%%% --------------------------------------------------------
%%% for tex4ht-html4.tex to update html4.4ht file 
%%% --------------------------------------------------------

\<configure html4 tabularray\><<<
\Configure{tabularray}{
  \ifvmode\IgnorePar\fi\EndP
  \gHAdvance\Next:TableNo by 1 \global\let\TableNo=\Next:TableNo%
  \HCode{<table class="tabularray \@currenvir" id="tbl-\TableNo">}
}{\ifvmode\IgnorePar\fi\EndP\HCode{</table>}}
{\HCode{<tr id="row-\TableNo-\HRow">}}{\HCode{</tr>}}
{\HCode{<td id="cell-\TableNo-\HRow-\HCol" style="\CellStyle" \CellAttributes>}}{\HCode{</td>}}
{\Configure{tabularrayattributes}{rowspan="\HRowspan"}}
{\Configure{tabularrayattributes}{colspan="\HMultispan"}}{}

% cells hidden by cell and row spans. they shouldn't be included in HTML, so we hide them using comments
\Configure{tabularrayignoredcell}{\HCode{<!-- ignored cell: }}{\HCode{ -->}}

% this is a default alignment, so we can ignore it, to save some space in the generated files
% \Configure{tabularrayhalign}{l}{\Configure{tabularraystyles}{text-align:left;}}
\Configure{tabularrayhalign}{r}{\Configure{tabularraystyles}{text-align:right;}}
\Configure{tabularrayhalign}{c}{\Configure{tabularraystyles}{text-align:center;}}
\Configure{tabularrayhalign}{j}{\Configure{tabularraystyles}{text-align:justify;}}

% this is a default alignment, so we can ignore it, to save some space in the generated files
% \Configure{tabularrayvalign}{m}{\Configure{tabularraystyles}{vertical-align:middle;}}
\Configure{tabularrayvalign}{h}{\Configure{tabularraystyles}{vertical-align:top;}}
\Configure{tabularrayvalign}{p}{\Configure{tabularraystyles}{vertical-align:top;}}
\Configure{tabularrayvalign}{f}{\Configure{tabularraystyles}{vertical-align:bottom;}}
% the vertical alignment of the following two is not correct, but CSS doesn't support the correct one
\Configure{tabularrayvalign}{t}{\Configure{tabularraystyles}{vertical-align:top;}}
\Configure{tabularrayvalign}{b}{\Configure{tabularraystyles}{vertical-align:bottom;}}

% Save column width only on the first row
\Configure{tabularraycolumnwidth}{\ifnum\HRow=1\Css{\#tbl-\TableNo\space td:nth-child(\HCol){width:\HColWidth;}}\fi}
\Css{table.tabularray{table-layout: fixed;border-collapse:collapse;margin:0.3em 0;}}
>>>
